%%%%%%%%%%%%%%% background %%%%%%%%%%%

\section{Background}
Particle distribution in parameter space is an important modeled quantity in
nuclear analysis. Particles (neutrons, photons, etc.) can vary in space and
energy, and system response functions are often extremely sensitive to these
parameter-space variations. Geometric meshes are used to describe spatial
distributions. Due to the nature of Monte Carlo statistical modeling, there
exists no analytic solution or numeric approximation to apply in geometric
space. As a result, particle track histories need to be tallied on a geometric
mesh.

Complex systems can be modeled very accurately using Monte Carlo simulations.
This degree of accuracy however is prequisite to an adequately-sampled geometry
with good statistics. These statistics are often improved with more particle
histories, which impacts not only the time required to generate the histories,
but the time to tally the histories on a mesh. Computation time becomes more of
for fine mesh grids over large geometric space. This is the motivation for mesh
tally accelerations.

Typically, this tallying process is performed sequentially by tracking the
particle through the mesh. A single CPU tracks the particles through the
geometry; it checks surface crossings, track lengths through voxels and updates
voxel position. Along the particle track, it's track length contributes to the
population of the mesh. All of the tallied histories provide a discritized
approximation of the particle distribution in parameter space.
