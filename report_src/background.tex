%%%%%%%%%%%%%%% background %%%%%%%%%%%

\section{Background}
Particle distribution in parameter space is an important modeled quantity in
nuclear analysis. Particles (neutrons, photons, etc.) can vary in space and
energy, and system response functions are often extremely sensitive to these
parameter-space variations. Geometric meshes are used to describe spatial
distributions. Due to the nature of Monte Carlo statistical modeling, there
exists no analytic solution or numeric approximation to apply in geometric
space. As a result, particle track histories need to be tallied on a geometric
mesh.

Complex systems can be modeled very accurately using Monte Carlo simulations.
This degree of accuracy however is prequisite to an adequately-sampled geometry
with good statistics. These statistics are often improved with more particle
histories, which impacts not only the time required to generate the histories,
but the time to tally the histories on a mesh. Computation time becomes more of
an issue for fine mesh grids over large geometric space. This is the motivation for mesh
tally accelerations.

The goal of tallying procedures is to convert the particle walk histories
to bins of particle fluence (volume-averaged) in discretized space (cartesian
voxels). This calculation is traditionally performed by CPUs and can be a
significantly large fraction of calculation time.
A single CPU tracks the particles through the
geometry; it checks surface crossings, track lengths through voxels and updates
voxel position. Along the particle track, it's track length contributes to the
population of the mesh. All of the tallied histories provide a discritized
approximation of the particle distribution in parameter space.

We used a toy scattering problem to test the effectiveness of GPU-computing for
mesh tallying. We created a sphere of arbitrary radius to test the mesh. Within
the sphere, we performed random walks with test particles. We sampled particle
tracks randomly in both direction and length as they scattered through and
(eventually) out of the sphere. All particles originated from the origin and
as they  walked through
the sphere, we recorded their scattering locations to be tallied later. Two
parameters were important to consider in the toy problem, sphere radius and mean
free path. The sphere radius determined how far particles could march before we
terminated their histories, and the mean free path impacted the average particle
path length in between scatterings.
