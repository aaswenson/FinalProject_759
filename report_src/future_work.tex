%%%%%%%%%%%%%%% Future work %%%%%%%%%%
\section{Future Work}
There are multiple oportunities for improvement of this work. On first outlook, 
the sequential part of this code is not handiling all the possible cases correctly. \
An example of this is related to the particle track, if a particle track starting 
outside the mesh, the algorithm set forth in \texttt{seq\_tally.cpp} will ignore this 
track and no contribution will be added to the flux. 
A problem arise if the particle track starts outside the mesh, but end inside the mesh, 
then the algorithm is falsely ignoring the track. This affects our final tallying of the 
flux. 


There are issues to be addressed with the parallel algorithm. 
From the analysis work, we see that the parallel algorithm is slower than the 
the sequential algorithm. The hypothesis is that this algorithm has a memory 
bound problem and a better understanding of memory usage is necessary to improve 
this algorithm. This is an important next step as it will help quantify if using GPU 
with the proposed algorithm is of interest.  

Another concern is related to mesh sizes. The algorithm segfaults if the mesh size 
combined with the mesh resolutions is too big. This issue might also be related to 
memory issues. 

Another improvement for this code could be to add other parameters that will help 
describe better the physics of the system. One important parameter is the 
energy of the particle. 

