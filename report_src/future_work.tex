%%%%%%%%%%%%%%% Future work %%%%%%%%%%
\section{Future Work}
There are multiple oportunities for improvement of this work. On first outlook, 
the sequential part of this code is not handiling all the possible cases correctly. \
An example of this is related to the particle track, if a particle track starting 
outside the mesh, the algorithm set forth in \texttt{seq\_tally.cpp} will ignore this 
track and no contribution will be added to the flux. 
A problem arise if the particle track starts outside the mesh, but end inside the mesh, 
then the algorithm is falsely ignoring the track. This affects our final tallying of the 
flux. 


There are also issues to be addressed with the parallel algorithm. The biggest one is that 
concerned with mesh sizes. The algorithm throwns a segfault if the mesh size combined 
with the mesh resolutions is too big. 

Another improvement for this code is to add other parameters that will help 
describe better the physics of the system. One important parameter is the 
energy of the particle. This will allow 
