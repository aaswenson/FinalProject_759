%%%%%%%%%%%%%%% sequential methods %%%%%%%%%%%%%%%%%%%%%%%%%%%%%%%%
\section{Sequential Methods}

The sequential code tracks each particle through a uniform, structured,
cartesian mesh. As each particle passes through mesh voxels, they contribute a
portion of their track length to the flux tally of the voxel. This process
requires at minimum three distance to crossing calculations per voxel. Equation
\ref{eqn:cross_plane} is the distance to plane crossing equation.

\begin{equation}
\label{eqn:cross_plane}
s = \frac{X_i - \vec{r}_i}{\vec{v}_i}
\end{equation}

Where $X_i$ is a normal plane in (x,y,z) space, $\vec{r}_i$ is the particle's
initial position, and $\vec{v}_i$ is the unit velocity vector.

Each voxel requires at minimum, three surface checks for crossing. Surfaces with
signs opposite of their respective particle velocity component can be neglected,
if the particle is traveling away from a surface, it will never cross that it.
Once the three eligible surfaces are selected, the minimum distance to crossing
is selected. 

The minimum distance to crossing is used to contribute particle track length to
the volume-averaged voxel flux. If the particle track is shorter than distance
to crossing, the contribution is adjusted accordingly.

The sequential code computes the track-length contribution on the mesh from
every particle history in serial. This is a computationally expensive process,
especially when particles have a large mean free path relative to the mesh
spacing. When this is the case, single tracks traverse a large number of mesh
points, each of which require (at minimum) three surface crossing calculations.


