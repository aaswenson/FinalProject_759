%%%%%%%%%%%%%%% Future work %%%%%%%%%%
\section{Future Work}
There are multiple oportunities for improvement of this work. Most importantly, 
the sequential part of this code is not correctly handling all particle-mesh
interation cases. The incorrect case involves a particle track starting 
outside the mesh, the algorithm applied in \texttt{seq\_tally.cpp} will ignore this 
track and no contribution will be added to the flux.
A problem arises if the particle track starts outside the mesh, but end inside the mesh, 
then the algorithm is falsely ignores the track. This affects the final flux
tally results.

There are also issues to be addressed with the parallel algorithm. 
From the analysis work, we show that the parallel algorithm is much slower than the 
the sequential algorithm. The hypothesis is that this algorithm is memory 
bound and a better understanding of memory usage is necessary to improve 
this algorithm. This is an important next step as it will help quantify if using GPU 
with the proposed algorithm is a worthwhile endeavor.  

Another concern is related to mesh sizes. The algorithm segfaults if the mesh size 
combined with the mesh resolutions is too big. This issue might also be related to 
memory issues.

Another improvement for the parallel code would add other parameters that
further describe the physics of the system. One important parameter is the 
energy of the particle. In physics simulation, particle energy is often
paramount to the behavior of flux response functions (like
cross-sections).

